\subsection{Model}
\label{sec:model}
\begin{figure}[h]
% \vspace{-1em}
\centering
\resizebox{0.55\linewidth}{!}{
\begin{tikzpicture}[->, >=stealth', auto, thick, node distance=1.3cm]
    % First figure
    \tikzstyle{every state}=[fill=white,draw=black,thick,text=black,scale=1,minimum size=0.7cm]
    \node[state]    (z)                                 {$z$};
    \node[state]    (z0)[above =0.6cm of z,label=center:$z_0$] {\phantom{$z$}};
    \node[state]    (x)[below left=0.48cm and 0.36cm of z,fill=black!20,label=center:$\tau$] {\phantom{$z$}};
    \node[state]    (y)[below right=0.48cm and 0.36cm of z,fill=black!20,label=center:$y$] {\phantom{$z$}};
    \path
    (z) edge[right]               (y)
    (z) edge[left]                (x)
    (z0) edge[left]      node{\small $z=U_\alpha{(z_0)}$}          (z);
    \node       (pa)[right= -0.05cm of z]    {\small $p_\alpha(z)$};
    \node       (px)[below= -0.08cm of x]    {\small $p_\beta(\tau|z)$};
    \node       (py)[below= -0.08cm of y]    {\small $p_\gamma(y|z)$};

    % Dashed line in the middle
    \draw[dashed, -] (1.75cm, -1.8cm) -- (1.75cm, 2.2cm);
    
    % Second figure
    \begin{scope}[shift={(4.5cm,1cm)}, block/.style={rectangle, draw, fill=black!10, minimum width=3cm, minimum height=0.6cm, align=center, rounded corners},
    arrow/.style={thin, -stealth'},
    bigbox/.style={draw, thick, rounded corners, inner sep=5pt, dashed}]
    
        % Sub-boxes
        \node[block] (cross) {\small Cross-attention};
        \node[block, below=0.75cm of cross] (causal) {\small Causal Transformer};

        % Main box that fits around the sub-boxes
        \begin{scope}[on background layer]
        \node[bigbox, fit=(cross) (causal)] (mainbox) {};
        \end{scope}

        % Times N at the top right of Causal Transformer
        \node[above right=0.1cm and -0.5cm of cross.north east, anchor=south west] (N) {$\times N$};

        % Variable z and arrows
        \node[left=0.5cm of mainbox.west] (z2) {$z$};
        \node[below=0.3 cm of mainbox] (xi) {\small$\tau=(s_1,a_1,s_2,a_2,\dots, s_H, a_H)$};
        \node[above=0.3 cm of mainbox] (xo) {\small$a_1,a_2,\dots,\cdots,a_H$};

        % Arrows to cross-attention
        \draw[arrow] (z2) -| ($(cross.south west)!0.25!(cross.south east)$);
        \draw[arrow] (z2) -| ($(cross.south west)!0.5!(cross.south east)$);
        \draw[arrow] (xi) -- (causal);
        \draw[arrow] (cross) -- (xo);

        % Arrow from causal transformer to cross-attention
        \draw[arrow] ($(causal.north west)!0.75!(causal.north east)$) -- ($(cross.south west)!0.75!(cross.south east)$);
    \end{scope}
\end{tikzpicture}
}
\caption{\textit{Left}: Overview of Latent Plan Transformer (LPT). $z\in\mathbb{R}^d$ is the latent vector. The prior distribution of $z$ is a neural transformation of $z_0$, i.e., $z = U_\alpha(z_0)$, $z_0 \sim {\cal N}(0, I_d)$. Given $z$, $\tau$ and $y$ are independent. $p_\beta(\tau|z)$ is the trajectory generator. $p_\gamma(y|z)$ is the return predictor.  \textit{Right}: Illustration of trajectory generator $p_\beta(\tau|z)$. }
\label{fig:LPT}
\end{figure}  

Given a variable-length trajectory $\tau$, $z \in \mathbb{R}^d$ is a vector that represents $\tau$ in the latent space. $y \in \mathbb{R}$ is the return of the trajectory. The joint distribution of the trajectory and its return is defined as $p(\tau,y)$.

The latent trajectory variable $z$, conceptualized as a plan, is posited to decouple the autoregressive policy and return estimation.
From a statistical standpoint, with $z$ given, we assume that $\tau$ and $y$ are conditionally independent, positioning $z$ as the information bottleneck. 
Under this assumption, the Latent Plan Transformer (LPT) can be defined as,
\begin{align}
p_{\theta}(\tau, y, z) &= p_{\alpha}(z) p_\beta(\tau|z) p_\gamma(y|z),
\label{eq:joint}
\end{align}
where $\theta = (\alpha, \beta, \gamma)$. 
LPT approximates the data distribution $p_\mathrm{data}(\tau, y)$ using the marginal distribution $p_\theta(\tau, y) = \int p_\theta(\tau, y, z) dz$. It also establishes a generation process,
\begin{align}
z \sim p_{\alpha}(z), \quad [\tau|z] \sim p_\beta(\tau|z), \quad [y|z] \sim p_\gamma(y|z).
\end{align}

The prior model $p_{\alpha}(z)$ is an implicit generator, defined as a learnable neural transformation of an isotropic Gaussian, $z = U_\alpha(z_0)$ and $z_0\sim \mathcal{N}(0,I_d)$.
$U_\alpha(\cdot)$ is an expressive neural network, such as the UNet~\citep{ronneberger2015u}. This approach is inspired by, yet contrasts with~\citet{pang2020learning}, wherein the latent space prior is modeled as an Energy-based Model (EBM) \citep{xie2016theory}. While EBM offers explicit unnormalized density, its sampling process is complex. Conversely, our model provides an implicit density with simpler sampling.

The trajectory generator $p_{\beta}(\tau|z)$ is a conditional autoregressive model with finite context $K$, $p_\beta(\tau|z) = \prod_{t=1}^H p_\beta(\tau_{(t)}|\tau_{(t-K)}, ..., \tau_{(t-1)}, z)$
where $\tau_{(t)}=(s_t, a_t)$. It can be parameterized by a causal Transformer with parameter $\beta$, similar to Decision Transformer~\citep{chen2021decision}. Specifically, the latent variable $z$ is included in trajectory generation using cross-attention, as shown in~\cref{fig:LPT} and controls each step of the autoregressive trajectory generation as $p_\beta(a_t|s_{t-K:t},a_{t-K:t-1}, z)$. The action is assumed to follow a single-mode Gaussian distribution, i.e. $a_t\sim \mathcal{N}(g_\beta(s_{t-K:t},a_{t-K:t-1}, z), I_{|A|})$. 

The return predictor is a non-linear regression on the latent trajectory variable $z$, modeled as $p_{\gamma}(y|z) = \mathcal{N}(r_\gamma(z), \sigma^2)$.
It directly predicts the final return from the latent variable $z$.
The function $r_\gamma(z)$ is a small multi-layer perceptron (MLP) that estimates $y$ based on $z$. The variance $\sigma^2$, is treated as the hyper-parameter in our setting. 

\subsection{Offline Learning}
\label{sec:offline_learning}

With a set of offline training examples $\{(\tau_i, y_i)\}_{i=1}^n$, we aim to learn Latent Plan Transformer (LPT) through maximum likelihood estimation (MLE). The log-likelihood function is defined as $L( \theta) = \sum_{i=1}^{n} \log p_\theta(\tau_i, y_i)$.
The joint probability of the trajectory and final return is
\begin{align}
   p_\theta(\tau, y) = \int p_{\beta}(\tau|z = U_\alpha(z_0)) p_{\gamma}(y| z = U_\alpha(z_0)) p_0(z_0) dz_0,
\end{align}
where $p_0(z_0)={\cal N}(0, I_d)$.
The learning gradient of log-likelihood can be calculated according to
\begin{align} 
\begin{split}
    \nabla_\theta  \log p_\theta(\tau, y)&=
    \E_{p_\theta(z_0|\tau, y)} [\nabla_\theta\log p_\beta(\tau|U_\alpha(z_0)) + \nabla_\theta\log p_\gamma(y|U_\alpha(z_0))].  
\label{eq:grad}
\end{split}
\end{align}
The full derivation of the learning method is in~\cref{appen:model}.
Let $\delta_\alpha, \delta_\beta, \delta_\gamma$ represent the expected gradients of $L(\theta)$ with respect to the model parameters $\alpha,\beta,\gamma$, respectively. The learning gradients for each component are formulated as follows.

For the prior model $p_\alpha(z)$,
\begin{align} 
% \label{eq:alpha}
  \delta_\alpha(\tau, y) &= \E_{p_\theta(z_0|\tau, y)} [\nabla_\alpha(\log p_\beta(\tau|z=U_\alpha(z_0))+ \nabla_\alpha\log p_\gamma(y|z=U_\alpha(z_0))]. \nonumber
\end{align}
For the trajectory generator,
\begin{align} 
% \label{eq:beta}
  \delta_\beta(\tau, y) = \E_{p_\theta(z_0|\tau, y)} [\nabla_\beta \log p_{\beta}(\tau|z=U_\alpha(z_0))], \nonumber
\end{align} 
For the return predictor,
\begin{align} 
% \label{eq:gamma}
  \delta_\gamma(\tau, y) = \E_{p_\theta(z_0|\tau, y)} [\nabla_\gamma \log p_{\gamma}(y|z=U_\alpha(z_0))]. \nonumber
\end{align} 

Estimating these expectations requires Markov Chain Monte Carlo~(MCMC) sampling of the posterior distribution $p_\theta(z_0|\tau,y)$. We use the Langevin dynamics~\citep{neal2011mcmc} for MCMC sampling, iterating as follows for a target distribution $\pi(z)$:
\begin{align} 
z^{k+1} = z^k + s \nabla_z \log \pi(z^k) + \sqrt{2s} \epsilon^k, 
 \label{eq:Langevin}
\end{align}
where $k$ indexes the time step of the Langevin dynamics, $s$ is the step size, and $\epsilon^k \sim {\mathcal N}(0, I_d)$ is the Gaussian white noise. Here, $\pi(z)$ is instantiated as the posterior distribution $p_\theta(z_0|\tau, y)$. We have $p_\theta(z_0|\tau, y)\propto p_0(z_0)p_\gamma(y|z)p_\beta(\tau|z)$, where $z=U_\alpha(z_0)$, such that the gradient is
\begin{align}
% \label{eq:p_z_ty}
    \nabla_{z_0} \log p_\theta(z_0|\tau, y) 
    =\nabla_{z_0}\underbrace{\log p_0(z_0)}_{\text{prior}}+\nabla_{z_0}\underbrace{\log p_\gamma(y|z)}_{\text{return prediction}}+\underbrace{\sum\nolimits_{t=1}^H\nabla_{z_0}\log p_\beta(\tau_{(t)}|\tau_{(t-K:t-1)},z)}_{\text{aggregating finite-context sub-trajectories}}. \nonumber
\end{align}
This demonstrates that the posterior inference of $z$ is an explicit process of optimizing a plan given its likelihood. In the presence of a finite context, $p_\beta(\tau|z)$ parametrized with Transformer can only account for sub-trajectories with a maximum length of $K$. The latent variable $z$ serves as an abstraction that integrates information from both the final return and sub-trajectories using gradients.

The sampling process starts by initializing $z_0^{k=0}$ from a standard normal distribution ${\mathcal N}(0, I_d)$. We then apply $N$ steps of Langevin dynamics (e.g., $N=15$) to approximate the posterior distribution, making our learning algorithm an approximate MLE. For a theoretical understanding of this noise-initialized finite-step MCMC, see \citet{pang2020learning,nijkamp2020learning2,XieZXL023}. 
However, for large horizons (e.g.,$H$=1000), this method becomes slow and memory-intensive. To mitigate this, we adopt the persistent Markov Chain (PMC)~\citep{tieleman2008training,xie2016theory,han2017abp}, which amortizes sampling across training iterations. During training, $z_0^{k=0}$ is initialized from the previous iteration and the number of updates is reduced to $N=2$ steps. See~\cref{appen:training} for training and architecture details.

\subsection{Planning as Inference}
\label{sec:inference}
The MLE learning of LPT gives us an agent that can plan. During testing, we first infer the latent $z_0$ given the desired return $y$ using Bayes' rule,
\begin{align}
    z_0\sim p_\theta(z_0|y) \propto p_0(z_0)p_\gamma(y|z=U_\alpha(z_0)).
\label{eq:p_z0_y}
\end{align}
This posterior sampling is achieved using Langevin dynamics similar to the training process. Specifically, we replace the target distribution in~\cref{eq:Langevin} with $p_\theta(z_0|y)$ and run MCMC for a fixed number of steps. Sampling from $p_\theta(z_0|y)$ eliminates the need for expensive back-propagation through the trajectory generator $p_\beta(\tau|z)$. 

This posterior sampling of $p(z_0|y)$ is an explicit process that iteratively refines the latent plan $z$, increasing its likelihood given the desired final return. It aligns with our intuition that planning is an inference process. This inferred $z$, fixed ahead of the policy execution, effectively serves as a plan. At each step, the agent consults this plan to generate actions conditioned on the current state and recent history, $a_t \sim p_\beta(a_t|s_{t-K:t-1},a_{t-K:t-1}, z=U_\alpha(z_0)).$

Once a decision is made, the environment's (possibly non-Markovian) transition $s_{t+1}\sim p(s_{t+1}|a_t, s_{t})$ emits the next state. This sequential decision-making process iterates the sampling of $s_t$ and $a_t$ until termination at the horizon.

\paragraph{Exploitation-inclined Inference (EI)}
Inspired by the classifier guidance (CG)~\citep{dhariwal2021diffusion,ho2022classifier} in conditional diffusion models, we introduce a guidance weight $w$ to the original posterior in \cref{eq:p_z0_y} 
\begin{align}
\label{eq:cg}
\Tilde{p}_\theta(z_0|y)\propto p_0(z_0)p_\gamma(y|z)^w, z=U_\alpha(z_0),
\end{align}
which has the score $\nabla_{z_0} \log \Tilde{p}_\theta(z_0|y)=\nabla_{z_0}{\log p_0(z_0)}+w\nabla_{z_0}{\log p_\gamma(y|z)}$.
This guidance weight $w$ controls the interpolation between exploration and exploitation. When $w=1$, the sampled plans collectively represent the posterior density and account for Bayesian uncertainty, resulting in a provably efficient exploration scheme~\citep{osband2017posterior}. When $w>1$, the sampled plans are more concentrated around the modes of the posterior distribution, which are plans more likely to the agent. The larger the value of $w$, the more confident the agent becomes, and the stronger the inclination towards exploitation.

An overview of the algorithms for both offline learning and inference can be found in the following.

\begin{algorithm}[h]
\caption{Offline learning}
\label{algo:learning}
\begin{algorithmic}
    \STATE {\bfseries Input:} Learning iterations~$T$, initial parameters~$\theta_0 = (\alpha_0, \beta_0, \gamma_0)$, offline training samples~$\mathcal{D}=\{\tau_i, y_i\}_{i=1}^n$, posterior sampling step size $s$, the number of steps $N$, and the learning rate $\eta_0,\eta_1,\eta_2$.
    \STATE {\bfseries Output:} $\theta_T$
    \FOR{$t=1$ {\bfseries to} $T$}
    \STATE{1.\textbf{Posterior sampling}}: 
    For each $(\tau_i, y_i)$, sample $z_0 \sim {p}_{\theta_t}(z_0|\tau_i, y_i)$ using~\cref{eq:Langevin} with $N$ steps and step-size $s$, where the target distribution $\pi$ is  ${p}_{\theta_t}(z_0|\tau_i, y_i)$. 
    \STATE{2.\textbf{Learn prior model} $p_\alpha(z)$, \textbf{trajectory generator} $p_\beta(\tau|z)$ and \textbf{return predictor} $p_\gamma(y|z)$}: \\
    $\alpha_{t+1} = \alpha_t + \eta_0 \frac1n\sum_i\delta_\alpha(\tau_i,y_i)$;
    $\beta_{t+1} = \beta_t + \eta_1 \frac{1}{n} \sum_{i}\delta_\beta(\tau_i,y_i)$;
    $\gamma_{t+1} = \gamma_t + \eta_2 \frac{1}{n} \sum_{i}\delta_\gamma(\tau_i,y_i)$ as in~\cref{sec:offline_learning}.
    \ENDFOR
\end{algorithmic}
\end{algorithm}
% \vspace{-1em}
\begin{algorithm}[h]
\caption{Planning as inference}
\label{algo:online}
\begin{algorithmic}
    \STATE {\bfseries Input:} Expected return $y$, a trained model on offline dataset $\theta$, posterior sampling step size $s$ and the number of steps $N$, Horizon $H$ and an evaluation environment.
    \STATE {\bfseries Output:} Trajectory $\tau$
    % \STATE \textit{\texttt{//Posterior sampling}} 
    \IF{Exploitation-inclined Inference (EI)}
    \STATE{Sample $z_0\sim \Tilde{p}_\theta(z_0|y)$ as in~\cref{eq:cg} using~\cref{eq:Langevin} with $N$ steps and step size $s$, where the target distribution $\pi$ is replaced by $\Tilde{p}_\theta(z_0|y)\propto p_0(z_0)p_\gamma(y|z=U_\alpha(z_0))^w$ and $z=U_\alpha(z_0)$.}
    \ELSE
    \STATE{Sample $z_0\sim p_\theta(z_0|y)$ as in~\cref{eq:p_z0_y} using~\cref{eq:Langevin} with $N$ steps and step size $s$, where $\pi$ is replaced by $p_\theta(z_0|y)\propto p_0(z_0)p_\gamma(y|z=U_\alpha(z_0))$ and $z=U_\alpha(z_0)$.}
    \ENDIF
    % \STATE{\textit{{Sample trajectory}}}
    \WHILE{current time step $t \le H$}
    \STATE Sample $a_t$ using trajectory generator as $a_t \sim p_\beta(a_t|s_{t-K:t-1},a_{t-K:t-1}, z=U_\alpha(z_0))$. \\Once a decision is made, the environment's transition $s_{t+1}\sim p(s_{t+1}|a_t, s_{t})$ emits the next state.\\
    \ENDWHILE
\end{algorithmic}
\end{algorithm}


