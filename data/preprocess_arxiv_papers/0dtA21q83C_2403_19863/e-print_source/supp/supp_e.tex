
\subsection{Limitations \& Broader Impact}
\label{supp_sec:limitations}

The primary challenge with this approach is the scalability issue when applied to a multi-bias setting. As the number of bias attributes increases, the subtle variations in linear decodability across the various branches could become so refined that accurately identifying biases may fail to achieve high fidelity. Moreover, depending on the network architecture might compel the model to depend excessively on intricate hyperparameter adjustments.

The societal impacts of identifying and mitigating biases in neural networks are extensive, resulting in fairer, more equitable, and trustworthy AI systems. Some of them are as follows :

\begin{enumerate}[leftmargin=*]
    \item Bias Mitigation in AI : contributes to more equitable AI systems by reducing the influence of spurious correlations.
    \item Societal Benefits: contributes to societal fairness by reducing biased decision-making in AI systems and potentially decreases the risk of discrimination in AI applications.
    \item Ethical AI Development: encourages transparency and accountability in AI research and deployment.
\end{enumerate}