\section{Conclusion}
\label{sec:conclusion}

We introduce DeNetDM, a novel debiasing method leveraging variations in linear decodability across network depths. Through extensive theoretical and experimental analysis, we uncover insights into the interplay between network architecture, attribute decodability, and training methodologies. DeNetDM employs paired deep and shallow branches inspired by the Product of Experts methodology, transferring debiasing capabilities to the desired architecture. By modulating network depths, it captures core attributes without explicit reweighting or data augmentation. Extensive experiments across various datasets, including synthetic ones like Colored MNIST and Corrupted CIFAR-10, as well as real-world datasets like Biased FFHQ and BAR, validate its robustness and superiority. Importantly, DeNetDM achieves performance comparable to supervised approaches, even without bias annotations. 
   
   
